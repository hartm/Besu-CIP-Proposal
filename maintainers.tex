%!TEX root = ./main.tex

\section{Mainnet Maintainers and Patrons} \label{sec:maintainers}
We now define \emph{maintainers} and \emph{patrons} of Besu and classify them appropriately.  An open-source \emph{contributor} is anyone who has contributed code or technical artifcacts (i.e. documentation) to a project.  A \emph{maintainer} is a contributor who has made substantial contributions to a project, to the point that the community has decided to promote them to a leadership role.  Maintainer responsibilities are varied across projects but generally include being one of the deciding votes on whether or not code is merged into the main branch of the codebase.  The requirements to become a maintainer also vary from project to project; for Besu, all things about maintainers are defined here~\cite{BesuM}.  

We emphasize that these are ``new'' definitions and not (to our knowledge) defined by the Ethereum community previously.

\subsection{Mainnet Maintainers}
We next define mainnet maintainers.  To do this, we first define what exactly constitutes mainnet work.  

\paragraph{Mainnet Work.} Below, we list some things that do and do not constitute mainnet work.  We note that the EF ultimately has full control over what exactly constitutes mainnet work and deciding what is or is not valid mainnet work is up to the EF alone.  However, we note that Ethereum mainnet development work includes (without limitation) contributions to features, performance, and stability required for Ethereum mainnet operation of the Besu client. This work includes supporting work necessarily deriving from this effort, such as build maintenance, code cleanup, documentation, test maintenance and development, and so forth. This also includes work on EIPs and appropriately submitted proposals intended for Ethereum Mainnet deployment, up until such a time as they are known to be excluded from Ethereum Mainnet deployment.

Some examples of work that is \emph{not} mainnet work would include work relating to Ethereum Classic development, GoQuorum compatibility, private transactions, private transaction enclaves, permissioning, etc., and supporting work deriving exclusively from these activities.  Work that may apply to both mainnet and non-mainnet applications (e.g. CI/CD expenses) is acceptable as long as it is not skewed towards non-mainnet applications.

\paragraph{Mainnet Maintainers.} Informally, a mainnet maintainer is an active maintainer (listed in the Besu maintainers.md file~\cite{BesuM}) whose contributions are primarily focused on Ethereum mainnet development or supporting Ethereum mainnet development.  It is expected that mainnet maintainers will make substantial contributions to mainnet development, including things like implementation of new features relevant to mainnet (e.g. sync, complex EIPs), significant performance improvements, and criticial security fixes.

While the above criteria is illustrative, it is informal.  Formally speaking, the EF will be the sole arbiter of who is classified as a mainnet maintainer.  At every grant period (see section~\ref{sec:defs}), the following process will occur:
\begin{itemize}
\item Any active Besu maintainer (listed in the maintainers list~\cite{BesuM}) can apply for mainnet maintainer status.  Any applicant who is not an active Besu maintainer will be immediately eliminated.
\item To begin, the HLF staff will create a file \emph{mainnet-maintainers-DATE.md} with \emph{DATE} denoting the start of the grant period in a Besu github directory decided by the Besu maintainers.  This will be done far enough in advance of a grant period to allow for a comfortable amount of time for both maintainers to apply and the EF to judge the applicants.
\item To apply, an active maintainer will list their name in the above file, as well provide links to their substantial mainnet contributions.
\item After a period (determined and announced by the EF), the EF will then mark each maintainer application as accepted or rejected in the .md file.  In order to encourage contriubtors that haven't met the standard, the EF may provide brief explanations of rejections.  The EF may also do this through private communications with the mainnet maintainers rather than in public.
\item The maintainers with applications accepted in this way will be recognized as the mainnet maintainers for that particular grant period.
\end{itemize}

The EF has the right to change the mainnet maintainer recognition process however they see fit, but they should notify the Besu community reasonably in advance if this is happening.

\subsection{Besu Patrons}
Individual contributors and organizations that have demonstrated a long-term pattern of substantial contributions to Besu may be categorized (by the EF) as \emph{patron organizations} or \emph{patron maintainers}.  As we explain below, the EF has the sole authority to determine patron status.

\paragraph{Patron Maintainer.}  A \emph{patron maintainer} is someone who the EF has identified as being a long-term, substantial contributor to mainnet development on Besu.  There is no formal process for becoming a patron maintainer.  Anyone wishing to become a patron maintainer should contact the EF and have a discussion with them.

\paragraph{Patron Organization.}  A \emph{patron organization} is an organization that the EF has identified as being a long-term, substantial contributor to mainnet development on Besu.  Like a patron maintainer, there is no formal process for becoming a patron organization, and this category exists at the discretion of the EF.

The initial patron organizations of the Besu CIP will be ConsenSys and the HLF.

\subsection{Company vs. Individual Maintainers}
Some Besu maintainers contribute on behalf of a company, and do their work on company time.  Others are either independent contractors or contribute in their own free time.  We refer to the former as \emph{company maintainers} and the latter as \emph{individual maintainers}.

While the HLF and Besu treat these classes of maintainers no differently, the Besu CIP will reward these classes of maintainers differently:  funds for company maintainers will be delivered to the company (which may or may not decide to give them to the maintainers), while funds for individual maintainers will go directly to the maintainer.  We explain this later in more detail.

