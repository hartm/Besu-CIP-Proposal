%!TEX root = ./main.tex

\section{Mainnet Maintainers and Patrons} \label{sec:maintainers}
We now define \emph{maintainers} and \emph{patrons} of Besu and classify them appropriately.  An open-source \emph{contributor} is anyone who has contributed code or technical artifcacts (i.e. documentation) to a project.  A \emph{maintainer} is a contributor who has made substantial contributions to a project, to the point that the community has decided to promote them to a leadership role.  Maintainer responsibilities are varied across projects but generally include being one of the deciding votes on whether or not code is merged into the main branch of the codebase.  The requirements to become a maintainer also vary from project to project; for Besu, all things about maintainers are defined here~\cite{BesuM}.  

\subsection{Mainnet Maintainers}
We next define mainnet maintainers.  To do this, we first define what exactly constitutes mainnet work.  

\paragraph{Mainnet Work.} Below, we list some things that do and do not constitute mainnet work.  We note that the EF ultimately has full control over what exactly constitutes mainnet work and deciding what is or is not valid mainnet work is up to the EF alone.  However, we note that Ethereum mainnet development work includes (without limitation) contributions to features, performance, and stability required for Ethereum mainnet operation of the Besu client. This work includes supporting work necessarily deriving from this effort, such as build maintenance, code cleanup, documentation, test maintenance and development, and so forth. This also includes work on EIPs and appropriately submitted proposals intended for Ethereum Mainnet deployment, up until such a time as they are known to be excluded from Ethereum Mainnet deployment.

Some examples of work that is \emph{not} mainnet work would include work relating to Ethereum Classic development, GoQuorum compatibility, private transactions, private transaction enclaves, permissioning, etc., and supporting work deriving exclusively from these activities.

\paragraph{Mainnet Maintainers.} Informally, a mainnet maintainer is an active maintainer (listed in the Besu maintainers.md file~\cite{BesuM}) whose contributions are primarily focused on Ethereum mainnet development or supporting Ethereum mainnet development.  It is expected that mainnet maintainers will make substantial contributions to mainnet development, including things like implementation of new features relevant to mainnet (e.g. sync, complex EIPs), significant performance improvements, and criticial security fixes.

While the above criteria is illustrative, it is informal.  Formally speaking, the EF will be the sole arbiter of who is classified as a mainnet maintainer.  At every grant period (see section~\ref{sec:defs}), the following process will occur:
\begin{itemize}
\item Any active maintainer (listed in the maintainers list~\cite{BesuM}) can apply for mainnet maintainer status.  
\end{itemize}

\textcolor{red}{TIM:  this is new text based upon our conversation and email exchange.  Does it look OK?}

\textcolor{green}{GRACE:  does this look OK from the perspective of the maintainers?}